% ************************** Thesis Acknowledgements *****************************

\begin{acknowledgements}      

First and foremost I want to acknowledge my supervisor Sarah Teichmann who has provided fantastic scientific mentorship. She let me plunge right into academic work, essentially learning by immersion. Sarah has had a surprising amount of time available for us students, in particular for meetings in person which often end up as great brainstorming sessions, resulting in many good ideas for analysis, experiments, or scientific questions to address.

Secondly I want to acknowledge Oliver Stegle for mentorship regarding probabilistic models and machine learning. After (intense) meetings with Oli I always learned something new and insightful.

The origin of my PhD needs to be acknowledged as well, with Luisa Hugerth and Carlos Talavera-Lopez, who (after a lot of nagging) convinced me to go into academic research and apply for the EMBL PhD programme. It was always great to meet up at times and go back to old habits of discussing fascinating science and having them teach me biology by answering all my ignorant questions, which is really all biology training I had before starting to work towards this thesis.

Over the years I have had some close collaborators who deserve special mention. In particular Kedar Natarajan who put in a lot of time and effort into performing many experiments for several studies. Ana Cvejic who cosupervised me for the thrombocyte development study, my first scientific project. Iain Macaulay, Charlotte Labalette who performed the sequencing and Zebrafish experiments for that study. I must thank Melanie Eckersley Maslin and Wolf Reik for an excellent collaboration, which did not make it into the thesis, but from which I learned a lot about developmental biology and epigenetics, as well as Ricardo Miragaia who performed the sequencing experiments for that study.

With Tapio Lonnberg, Kylie James, and Ashraful Haque I had a great collaboration for the immune cell differentiation study, which is the work in this thesis I'm the most proud of. Even though problems always happen with single cell RNA-sequencing, data from Tapio have a tendency to be of very high quality.

Lam-Ha Ly very efficiently provided invaluable help finalizing several projects, as well as brightening the days for us all in the brief time she was working with us.

I am grateful for the great EMBL Predoc course, both for everything I learned, and for getting to know the EMBL scientific community.

Over the year I have had several desk neighbors, but when I first started Xiuwei Zhang was great for discussing my statistics and analysis questions or ideas.

The other Teichmann lab members and alumni deserve acknowledgement for helping me towards this thesis in many ways, but I will particularly mention Michael Stubbington, Valentina Proserpio, Xi Chen, Tzachi Hagai, Tomas Gomes, Aik Ooi, Mirjana Efremova, and Roser Vento-Tormo for a lot of great discussions about biology, technology, analysis, and everything between.

Finally, and most importantly, I want to acknowledge Zheng-Shan Chong for all the good times and discussions. For all the good questions she asked which made me think clearer about my projects. All insight, knowledge and inspiration. Without Shan I would have had a miserable time during my PhD, and thanks to her I have learned a lot.

\end{acknowledgements}
