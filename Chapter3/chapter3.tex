%*******************************************************************************
%****************************** Third Chapter **********************************
%*******************************************************************************

\chapter{\name{Pseuotime} analysis reveals a continuous spectrum of differentiation in haematopoietic cells} \label{ch:zebrafish}

\graphicspath{{Chapter3/Figs/}}

The transcriptional programs that govern hematopoiesis have been investigated primarily by population level analysis of hematopoietic stem and progenitor cells, which cannot reveal the continuous nature of the differentiation process. Here we applied single-cell RNA-sequencing to a population of hematopoietic cells in zebrafish as they undergo thrombocyte lineage commitment. By reconstructing their developmental chronology computationally, we were able to place each cell along a continuum from stem cell to mature cell, refining the traditional lineage tree. The progression of cells along this continuum is characterised by a highly coordinated transcriptional program, displaying simultaneous suppression of genes involved in cell proliferation and ribosomal biogenesis as the expression of lineage specific genes increases. Within this program, there is substantial heterogeneity in the expression of the key lineage regulators. Overall, the total number of genes expressed, as well as the total mRNA content of the cell, decreases as the cells undergo lineage commitment.

The work in this chapter was published in \textit{Cell Reports} as \textit{Single cell RNA-sequencing reveals a continuous spectrum of differentiation in haematopoietic cells} \cite{Macaulay2016-zd}. The same analysis procedure was also applied to study mESC development in a study which was published as \textit{MERVL/Zscan4 Network Activation Results in Transient Genome-wide DNA Demethylation of mESCs} in \textit{Cell Reports} \cite{Eckersley-Maslin2016-cz}.

\subsection*{Individual contributions}

The study in this chapter was designed by Ana Cvejic, who interpreted the results and was the senior author of the manuscript. The zebrafish experiments were performed by Charlotte Labalette, and single cell RNA-seq experiments were performed by Iain Macaulay; both of whom are co-first authors of the manuscript. For detailed experimental procedures, see Section \ref{sec:zebrafish-methods}.

The first part of the chapter describes the experimental study which motivated making a model for \name{Pseudotime}. After the experimental results, we give an in depth description of the models we are using, in the context of single cell gene expression.

\section{Introduction}

Hematopoietic stem cells (HSCs) have the ability to self-renew and give rise to all different blood cell types \cite{Orkin2008-os}. Our understanding of the functional properties of different hematopoietic cell types has been advanced mainly by population level analysis. Current methods of purifying hematopoietic cells to relative homogeneity are based on the expression of specific combinations of cell surface markers. However, a homogeneous population of cells, as determined by a well-defined set of cell surface markers, may include many functionally distinct populations. This was nicely illustrated in studies showing that within the HSC compartment, individual HSCs may have differ in their ability to reconstitute the blood cell lineages (e.g., balanced production of myeloid and lymphoid cells or deficiency in lymphoid potential) \cite{Muller-Sieburg2012-vb}. More recently, it was demonstrated that common myeloid progenitors (CMP) are a mixed population of cells with distinct lineage potentials \cite{Notta2015-ng}. The lack of CMPs as a separate cell entity with broad myeloid potential brings into question the traditional model of hematopoietic lineage development and further underscores the importance of revising the current view of lineage development in hematopoiesis. Therefore, there is a need to address the exact composition of the stem and progenitor populations \textit{in vivo}, as well as the relationships between them. Single cell transcriptome analysis might provide answers to these outstanding questions \cite{Cvejic2015-ra}.

Among vertebrate models, the zebrafish provides a unique combination of advantages for the study of blood development at the single cell level. Zebrafish blood contains cells of all hematopoietic lineages and orthologs of most transcription factors involved in mammalian hematopoiesis have been defined \cite{Hsia2005-fg, Song2004-jn}. Importantly, transcriptional mechanisms and signaling pathways in hematopoiesis are well conserved between zebrafish and mammals, making them a clinically relevant model system \cite{Jagannathan-Bogdan2013-du}.

Over the past few years, a number of transgenic zebrafish lines have been generated in which promotors specific to hematopoietic subpopulations drive expression of fluorescent molecules \cite{Carradice2008-lz}. These reporter lines provide a valuable resource of labelled cells ranging from HSCs to a wide range of mature blood cell types. As in mammals, adult hematopoiesis in zebrafish is both continuous and asynchronous. Thus, a single sample of kidney marrow (the analogous tissue to mammalian bone marrow) contains the full spectrum of hematopoietic cell types at various stages of differentiation at any one time. As this is the single site of hematopoiesis in zebrafish, and is easily accessible, the cells are minimally perturbed when sorted \textit{ex vivo}, making this an ideal system to study basic principles of regulation of differentiation, both at the molecular and cellular levels.

Here we used high-throughput single-cell RNA sequencing combined with fluorescence-activated cell sorting (FACS) index sorting\footnote{\textit{Index sorting} refers to maintaining flow cytometric analysis values for the individual cells sorted into wells.} of adult zebrafish marrow-derived hematopoietic cells. We ordered cells by their progression through differentiation based on gene expression profiles using no prior knowledge of which cell population they belong to. Our analysis revealed the continuous nature of thrombocyte development and the coordinated transcriptional programs that govern cell differentiation. Interestingly, thrombocytes in zebrafish remain transcriptionally active even after leaving the kidney marrow and entering the circulation.

\section{Profiling individual hematopoietic cells \textit{ex vivo}}

Here, we used single-cell RNA-sequencing (RNA-seq) of zebrafish kidney cells to resolve the cellular hierarchy of lineage development in the myeloid branch of hematopoiesis. To focus on this lineage, we used expression of CD41 as a marker of HSCs and the megakaryocyte equivalent in fish (``thrombocytes''). CD41 in human is highly regulated during hematopoietic development  \cite{Debili2001-ib, Robin2011-en}, with CD41 being switched on. In zebrafish the Tg(cd41:EGFP) reporter line labels two distinct populations of cells that express the cd41-EGFP transgene. The weakly fluorescent (EGFP\textsuperscript{low}) subset marks HSCs and progenitor cells \cite{Ma2011-hg} and the brightly fluorescent (EGFP\textsuperscript{high}) subset includes mature and differentiated thrombocytes \cite{Ma2011-hg}.

\begin{figure}
    \centering
    \includegraphics[width=0.9\textwidth]{"Figure1"}
    \caption[cd41 cells transition through five transcriptional states during thrombocyte differentiation in zebrafish]{\textbf{cd41 cells transition through five transcriptional states during thrombocyte differentiation in zebrafish.} (A) A single kidney, from a heterozygote Tg(cd41:EGFP) reporter fish, was dissected and carefully passed through a strainer. Using flow cytometry, EGFP\textsuperscript{low} and EGFP\textsuperscript{high} cells were identified and 188 cells from each population were index sorted. Two wells (in red) per plate were left without cells. RNA from each cell was isolated and used to construct a single mRNA-seq library per cell, which was then sequenced using Hi-seq. (B)	t-SNE plot of the RNA-seq data from 363 EGFP\textsuperscript{low} and EGFP\textsuperscript{high} cells. (C)	The same t-SNE plot (as shown in B) but with points colored based on the cluster the cells belong to. Clusters are labeled as 1a, 1b, 2, 3, 4, and outlier cells. See also Figures \ref{fig:gating}, \ref{fig:qc}, and \ref{fig:ica}.}
    \label{fig:sorting}
\end{figure}

\begin{sloppypar}
Using flow cytometry, we identified EGFP\textsuperscript{low} and EGFP\textsuperscript{high} cells and sorted 188 cells from each population from a single kidney from a Tg(cd41:EGFP) reporter fish (Figure \ref{fig:sorting}A; Figures \ref{fig:gating}A-\ref{fig:gating}I). Each EGFP+ cell was collected in a single well of a 96-well plate, and for each cell, its size (FSC), granularity (SSC), and EGFP fluorescence level were recorded. Single-cell mRNA-seq libraries were constructed and sequenced to a depth of around 2.5 million reads per library. Of 376 cells, 13 cells failed our quality control (QC) and were removed from further analysis (Section \ref{sec:zebrafish-methods}; Figures \ref{fig:qc}A and \ref{fig:qc}B). For the remaining 363 cells, we accurately quantified between \( 1,000 \) and \( 6,000 \) genes per cell.
\end{sloppypar}

\section{Ordering hematopoietic cells from a single kidney across lineage development}

\begin{sloppypar}
To identify groups of cells and order them in terms of their developmental progression, we initially used a multi-step approach. First, we used independent component analysis (ICA) to identify distinct factors that describe the variability of EGFP cells. ICA revealed four latent factors (hidden variables) that explain (1) a progression among EGFP\textsuperscript{low} cells (‘‘within\_small\_component’’), (2) a switch from EGFP\textsuperscript{low} cells toward EGFP\textsuperscript{high} cells (‘‘difference\_component’’), and (3) progression among the EGFP\textsuperscript{high} cells (‘‘within\_large\_component’’). Finally, the fourth factor identified three outlier cells (‘‘outlier\_component’’) (Figure \ref{fig:ica}A).
\end{sloppypar}

To facilitate data depiction, we used non-linear dimensionality reduction (t-distributed stochastic neighbour embedding [t-SNE];  \cite{Van_der_Maaten2008-lh}) to represent the four latent factors in two dimensions (Figure \ref{fig:sorting}B). ICA revealed a clear distinction between EGFP\textsuperscript{low} and EGFP\textsuperscript{high} cells, implying sharp divergence at the transcriptional level (Figure \ref{fig:ica}A; Figure \ref{fig:sorting}B).

In addition, EGFP\textsuperscript{low} cells are a more heterogeneous group compared to EGFP\textsuperscript{high} cells. To explore this further, we used hierarchical clustering to partition EGFP cells based on their independent components (Figure \ref{fig:ica}B). Interestingly, whereas EGFP\textsuperscript{low} cells were split into four distinct clusters (here named 1a, 1b, 2, and 3), EGFP\textsuperscript{high} cells were all grouped into a single cluster (here named 4), confirming the substantial heterogeneity of the EGFP\textsuperscript{low} population of cells (Figure \ref{fig:sorting}C).

Differentiation of hematopoietic cells involves the acquisition of specific phenotypes that depend on the repression of genes characteristic of a multipotent cell state and expression of lineage-restricted genes  \cite{Seita2010-mk}. Thus, the whole process can be conceptualized as a temporal ordering of a highly coordinated transcriptional program through which each cell progresses. To examine the transcriptional transitions undergone by cd41-EGFP cells during differentiation, we ordered cells based on the cluster they belonged to, the latent factor that explains the variability of the cells within the cluster, and the level of EGFP fluorescence (details provided in the Experimental Procedures). Our model assumes gradual changes in gene expression during developmental progression of thrombocytes along a one-dimensional (i.e., non-branching) path. (We could not detect any apparent branch point in the data.) This ranking of cells through the entire process was treated as ``pseudotime.''

\begin{figure}
    \centering
    \includegraphics[width=\textwidth]{"Figure2"}
    \caption[Ordering of cells through the developmental trajectory]{\textbf{Ordering of cells through the developmental trajectory.} (A)	We inferred a smooth progression over the developmental lineage, represented as pseudotime, using two different methods. Here we demonstrate the path of both pseudotimes by regressing them into a t-SNE plot of the data. Points are colored based on the cluster the cells belong to. (B)	The pseudotime inferred with two different methods correlate very strongly (Spearman correlation 0.97). (C)	Expression of cd41 mRNA (top), GFP mRNA (middle), and GFP fluorescence (bottom) shown in pseudotime. Each point represents an individual cell; points are colored based on the cluster the cells belong to. See also Figure \ref{fig:ica}.}
    \label{fig:pseudotime}
\end{figure}

To ensure our pseudotime ordering was stable, we also ordered the cells using a more principled method, a Bayesian Gaussian Process Latent Variable Model (BGPLVM) \cite{Titsias2010-hq} (see Section \ref{sec:zebrafish-methods}). Comparing the paths these orderings take when regressed into the t-SNE depiction, one can appreciate the similarity between them (Figure \ref{fig:pseudotime}A). The two pseudotime orderings agreed very strongly (Spearman correlation 0.97; Figure \ref{fig:pseudotime}B), giving us confidence in our ordering.

When presented in pseudotime, the expression of endogenous cd41 (also known as itga2b) and EGFP, as well as EGFP fluorescence, recorded during sorting, were highly correlated and showed an expected increase through pseudotime (Spearman rho 0.85, 0.80, and 0.82, respectively) (Figure \ref{fig:pseudotime}C). This supports our pseudotime ordering of the cells from the HSC to the differentiated thrombocyte extracted from a single kidney.

\section{Inferring cell states in the myeloid lineage}

\begin{figure}
    \centering
    \includegraphics[width=\textwidth]{"Figure3"}
    \caption[Expression of key regulators of hematopoiesis over \name{pseudotime}]{\textbf{Expression of key regulators of hematopoiesis over \name{pseudotime}.} Expression (in TPM) of genes, relevant in hematopoiesis, over pseudotime. Points are colored based on the cluster the cells belong to. For each cluster, we show the proportion of cells within the given cluster expressing the gene at TPM > 1. HSC, hematopoietic stem cells-affiliated genes; Meg-Erythroid, megakaryocyteerythroid progenitors-affiliated genes; Myeloid, myeloid lineage-affiliated genes.}
    \label{fig:genes}
\end{figure}

To define the identity of cell types within the five clusters, we evaluated the expression of orthologs of transcription factors and other genes known to be relevant in mammalian hematopoiesis, including the expression of early (cd61, also known as itgb3a/b) and late (cd42b, also known as gp1bb) markers of megakaryocyte differentiation (Figure \ref{fig:genes}). The panel of genes analyzed was representative of HSCs (Tal1, Lmo2, Lyl1, Gata2, Runx1, Meis1, C-myb, and Erg)  \cite{Capron2006-mp, Greig2008-ab, Loughran2008-el, Orkin2008-os, Pineault2002-kl}, megakaryocyte/erythroid (Fli1, Gfi1b, Gata1, Cd61, Cd42b, Vwf, and Selp)  \cite{Clay2001-fa, Orkin2008-os, Poirault-Chassac2010-bf, Schick1993-lj}, and myeloid- (Gfi1, Pu.1 also known as spi1a/b, and Cebp1)  \cite{Tenen1997-dr, Zeng2004-dh} lineage-affiliated genes.

For each gene, we assessed the level of its expression in pseudotime, as well as the fraction of cells that expressed the gene of interest in each of the clusters (Figure \ref{fig:genes}). For example, c-myb was highly expressed in cluster 1a, as well as in clusters 1b, 2, and 3, but was downregulated in cluster 4. This is in line with previous reports that C-myb is expressed in immature hematopoietic cells and is downregulated during differentiation  \cite{Greig2008-ab}. Cells in cluster 1a had relatively high expression of lmo2, tal1, and meis1. These genes, together with fli1, showed a similar distribution of expression across pseudotime, whereas gata2 was more restricted to cluster 1a. The mammalian HSC genes runx1 and erg were expressed at a relatively low level overall, and in a small fraction of cells within all clusters. Overall, most of the mammalian HSC marker genes examined are expressed in cluster 1a, and to a lesser degree in 1b, 2, and 3.

In contrast, Gata1 and Gfi1b are known to be expressed at high levels in the erythroid and megakaryocyte lineages  \cite{Orkin2008-os, Vassen2007-ul} but not in HSCs. In our dataset, gata1a and gfi1b were expressed in all clusters except cluster 1a. Furthermore, expression of both early (itgb3a/b) and late (gp1bb) markers of megakaryocyte differentiation started very early and peaked late in pseudotime, confirming that more mature thrombocytes are largely confined to cluster 4.

We also assessed the expression of two well-known platelet genes, vWf (von Willebrand factor) and selp (P-selectin), through pseudotime (Figure \ref{fig:genes}). Our analysis revealed that, contrary to previous reports  \cite{Carrillo2010-gy}, thrombocytes in zebrafish do not express von Willebrand factor and P-selectin. This was confirmed by qPCR analysis of cd41 EGFP\textsuperscript{high} thrombocytes from zebrafish kidney. We found, however, that vWf was ex- pressed in the whole kidney sample and in fli1:GFP positive cells sorted from Tg(fli:EGFP) fish, suggesting that the vWf expression pattern differs somewhat between zebrafish and mammals.

Surprisingly, myeloid lineage-affiliated genes (e.g., spi1, gfi1, and cebp1) were largely absent across all cells (Figure \ref{fig:genes}). This suggests that there is no common myeloid progenitor population in this dataset, which charts a continuous HSC to thrombocyte pathway. Altogether, our data are consistent with the notion that cells from cluster 1a belong to HSCs that transition directly to erythroid-thrombocyte progenitor cells, possibly circumventing the CMP step. Although this is surprising, there are other reports of direct, unconventional, HSC to megakaryocyte-erythroid progenitor transitions, such as a recent report in mouse \cite{Guo2013-si}.

Identification of these intermediate progenitor and differentiated cell types prompted us to carry out additional analyses of the sets of genes that strongly correlate with individual cell types. We used a machine learning method, random forest feature importance, to find genes whose expression ‘‘marks’’ distinct clusters of cells. The unique sets of genes expressed in each of the cell types provide an opportunity to reveal novel markers of the identified cell types, and at the same time, provide more insight into their biological function.

\begin{figure}
    \centering
    \includegraphics[width=\textwidth]{"Figure4"}
    \caption[Identification of new cell-type markers]{\textbf{Identification of new cell-type markers.} (A) t-SNE plot of the RNA-seq data from 363 EGFP cells. Points are colored based on the cluster the cells belong to. Selected genes, whose expression is highly correlated with individual clusters, are shown next to each cluster. Selected gene ontology terms associated with genes that are highly correlated with cluster 2 and the outlier cells are included. (B) Expression of marker genes over pseudotime (left). Points are colored based on the cluster the cells belong to. For each cluster, we show the proportion of cells expressing the gene at TPM > 1. Expression of pairs of genes is shown on the right. Points are colored based on the cluster the cells belong to. The side diagrams show the proportion of cells within the cluster expressing the gene at the given level of expression. (C) Cell cycle analysis of three different populations of EGFP cells. The GFP\textsuperscript{low}SSC\textsuperscript{high} cells are enriched for cells from clusters 1a/1b/2, GFP\textsuperscript{low}SSC\textsuperscript{high} and GFP\textsuperscript{high} cells are enriched for cells from clusters 3 and 4, respectively. An average of two experiments is shown as a percentage of cells in G0 and G1 (G0/1) and S and G2 phase (S/G2) ± SEM. (D) Distribution of FSC (top) and SSC (bottom) values in the different clusters. In particular, one can see that the small population of outliers (cluster x, shaded gray) has higher FSC and SSC values than cells from other clusters. See also Figures \ref{fig:newgate} and \ref{fig:cytospin}.}
    \label{fig:clusters}
\end{figure}

Among the numerous newly identified cell-type markers, we found several of particular interest (Figure \ref{fig:clusters}A). Ccr9a is a member of the beta chemokine receptor family and is known to be expressed in HSCs \cite{Wright2002-xi}; our data show that ccr9a expression is highly correlated with cluster 1a (Figure \ref{fig:clusters}B). Transcription elongation factor A (SII), tcea3, was specifically expressed in cluster 1b (Figure \ref{fig:clusters}B). Cells from cluster 1b can also be sorted by combining expression of plasminogen receptor gene (plgrkt) and ascc1 (Figure \ref{fig:clusters}B). Good marker genes for cluster 2 included e2f8, which encodes a protein involved in progression through the cell cycle \cite{Deng2010-am} and top2a, a DNA topoisomerase involved in processes such as chromosome condensation and chromatid separation  \cite{Downes1994-sl} (Figure \ref{fig:clusters}B). Interestingly, the enriched gene ontology (GO) terms for cluster 2 included cell division and cell cycle (Figure \ref{fig:clusters}A), suggesting that an expansion phase precedes lineage commitment and terminal differentiation of thrombocytes.

To experimentally validate the prediction of greater proliferation in this progenitor population, we sorted cells from clusters 1a/1b/2 versus 3 and 4, based on EGFP fluorescence, and SSC and FSC (Figures \ref{fig:newgate}A–\ref{fig:newgate}G). We compared the cell cycle distributions of the sorted populations using propidium iodide (PI) staining. The combined cells from clusters 1a/1b/2 had a significantly higher proportion of cells in S and G2/M phase compared to clusters 3 and 4 (Figure \ref{fig:clusters}C), validating our finding that these cells proliferate faster.

These results show that expression of EGFP together with SSC and FSC values could be used to efficiently separate cells from clusters 3 and 4 from the early progenitor populations (1a/1b/2) in the cd41 reporter line (Figures \ref{fig:newgate} and \ref{fig:cytospin}). Additional markers for cluster 3 included combined high expression of fzd8b and no expression of mibp (Figure \ref{fig:clusters}B). For cluster 4, a high level of cd41 uniquely marks this population.

Finally, we also assessed a unique set of genes expressed by the three outlier cells. GO enrichment analysis of their marker genes yielded only three statistically significant GO terms, all linked with immunity (Figure \ref{fig:clusters}A). One plausible explanation is that these outlier cells represent macrophages that have engulfed or are attached to thrombocytes and hence retained a high level of EGFP fluorescence. Indeed, the outlier cells expressed an array of macrophage/monocyte affiliated genes such as mpeg (macrophage expressed gene 1), csf1r (colony-stimulating factor 1 receptor), csf3r (colony-stimulating factor 3 receptor) etc. Furthermore, compared to all other cells, the outlier cells had remarkably high FSC and SSC values, characteristic of macrophages (Figure \ref{fig:clusters}D).

\section{Validation of developmental progression from the kidney and circulation}

\begin{figure}
    \centering
    \includegraphics[width=\textwidth]{"Figure5"}
    \caption[Validation of identified early clusters and terminal state of late cluster]{\textbf{Validation of identified early clusters, and terminal state of late cluster.} (A–C) In a second experiment, only cells belonging to the early clusters were sorted. The distributions of (A) cd41 expression, (B) number of expressed genes, and (C) endogenous mRNA content of cells are as expected in the populations of cells sorted from kidney. (D) mRNA expression of cd41 in the sorted populations of cells. We see the expected increase from Kidney EarlyEnriched through Kidney EGFP\textsuperscript{low} to finally Kidney EGFP\textsuperscript{high}. Expression of cd41 did not change between Kidney EGFP\textsuperscript{high} and EGFP\textsuperscript{high} cells in circulation (likelihood ratio test, p = 1 after correcting for multiple testing.) (E and F) When developing from EarlyEnriched through EGFP\textsuperscript{low} to EGFP\textsuperscript{high}, the cells express fewer genes and contain less mRNA, confirming the pseudotime ordering we observed in the initial experiment. There was no change in the number of expressed genes and RNA content between kidney- and circulation-derived EGFP\textsuperscript{high} cells. See also Figures \ref{fig:rep-qc} and \ref{fig:technical}.}
    \label{fig:replicate}
\end{figure}

Having identified the developmental progression in our initial experiment, we set out to validate our findings in a second fish, examining both kidney and circulating cells. We sorted an additional 92 cells from cluster 1a/1b/2 (named here EarlyEnriched), 46 EGFP\textsuperscript{low} cells and 46 EFPhigh cells from the kidney of another Tg(cd41:EGFP) fish. We also sorted 24 EGFP\textsuperscript{low} and 68 EGFP\textsuperscript{high} circulating cells from the same fish (Figure S6A). Our analysis confirmed that the pattern of ICA follows the same structure as observed in the previous experiment (Figure S6B). This means that the cell populations and their relative relationships are conserved in this biological replicate. Similarly, the pseudotime ordering of EarlyEnriched, EGFP\textsuperscript{low}, and EGFP\textsuperscript{high} cells in the kidney recapitulated patterns we identified in the initial experiment (Figures \ref{fig:replicate}A–\ref{fig:replicate}F).

In addition, we discovered that EGFP\textsuperscript{high} cells in circulation are transcriptionally identical to EGFP\textsuperscript{high} cells in the kidney, with no significant change in the number of expressed genes (Figure \ref{fig:replicate}B), RNA content (Figure \ref{fig:replicate}C), or any gene’s expression pattern (likelihood ratio test, corrected for multiple testing with Holm-Sidak). We concluded, therefore, that the thrombocytes exit the kidney in a fully mature state and are maintained in a transcriptionally active state in circulation.

In both datasets, the total number of genes and total mRNA content expressed per cell were correlated with its differentiation state (Figure \ref{fig:technical}). This was not due to a difference in the sequencing depth or cell size (Figure \ref{fig:technical}). Instead, it represents a biological difference between cells during development. This supports the idea that more differentiated, post-mitotic cells (clusters 3 and 4) have a specialized transcriptional program with expression of a small, focused set of genes (Figure \ref{fig:technical}).

\section{Transcriptional modules related to growth and proliferation in the thrombocyte developmental gene expression program}

\begin{figure}
    \centering
    \includegraphics[width=\textwidth]{"Figure6"}
    \caption[Identification of genes that are dynamically regulated over \name{pseudotime}]{\textbf{Identification of genes that are dynamically regulated over \name{pseudotime}.} (A)	Pseudotime expression patterns of genes (rows) that significantly vary over pseudotime progression (X axis). Every row is the Z score scaled Gaussian process regression prediction representing the expression pattern. (B)	The gene expression pattern for the underlying function explaining the expression pattern in each group is shown as a black line (95\% confidence interval in the gray area). Below, selected gene ontology terms associated with the genes in each group are shown. (C)	Expression (in TPM) of an example gene from each group through pseudotime. Points are colored based on the cluster the cells belong to.}
    \label{fig:mohgp}
\end{figure}

To find genes with similar trends in expression across pseudotime, we used a mixtures of hierarchical Gaussian processes model to cluster the pseudotime series \cite{Hensman2015-op}. We identified 130 genes that are dynamically expressed through pseudotime. Clustering of these genes revealed three distinct patterns of their progression during differentiation (Figure \ref{fig:mohgp}A. Genes upregulated early in pseudotime and then downregulated later (group I) were significantly enriched with the GO term ‘‘nucleic acid binding’’ and ‘‘chromosome maintenance’’ (Figures \ref{fig:mohgp}B and \ref{fig:mohgp}C, possibly reflecting the increased proliferation of cells earlier in pseudotime. Genes gradually downregulated through pseudotime (group II) were highly enriched with the GO terms ‘‘eukaryotic translation elongation’’, ‘‘ribosomes’’ etc. (Figures \ref{fig:mohgp}B and \ref{fig:mohgp}C. Expression of these genes was highly correlated with the general trend of decreased RNA content over pseudotime (Spearman rho = 0.85), therefore suggesting a regulatory loop between the total RNA content in the cell and expression of genes that encode proteins relevant for ribosome synthesis. Finally, genes upregulated early and then maintained at a high level (group III) were highly enriched with the GO terms ‘‘ECM-receptor interaction,’’ ‘‘platelet aggregation,’’ and ‘‘hemostasis,’’ pointing to genes important for thrombocyte function (Figures \ref{fig:mohgp}B and \ref{fig:mohgp}C(. Taken together, our analysis suggests that differentiation of thrombocytes is governed by coordinated transcriptional programs that limit the proliferation of cells and their translational capacity while simultaneously promoting expression of genes relevant to thrombocyte function.

\section{Single cell gene expression patterns of whole-genome duplicated genes}

\begin{figure}
    \centering
    \includegraphics[width=\textwidth]{"Figure7"}
    \caption[Single cell analysis reveals three main patterns of usage of duplicated genes during thrombopoiesis in zebrafish]{\textbf{Single cell aAnalysis reveals three main patterns of usage of duplicated genes during thrombopoiesis in zebrafish.} (A) Ohnolog gene pairs were divided into four classes based on thresholds in a decision tree. (B) Expression (in TPM) of example ohnologs, randomly selected from each class, in individual cells. Points are colored based on the cluster the cells belong to. XOR ohnolog: both ohnologs are expressed but never in the same cell. Single ohnolog: just one ohnolog is expressed. Mixed ohnologs: both ohnologs are expressed in individual cells.}
    \label{fig:ohnologs}
\end{figure}

Gene duplication is a common event in eukaryotic genomes \cite{Meyer1999-lb} and due to the teleost-specific genome duplication around 26\% (i.e., \( 3,440 \)) \cite{Howe2013-ul} of zebrafish genes are duplicated. Gene duplicates that originate from genome duplication are called ohnologs. To assess the use of duplicated genes during thrombopoiesis in zebrafish, we examined the expression of ohnologs in each of the 363 cd41:EGFP cells. Of \( \sim 8,000 \) ohnolog categories \cite{Howe2013-ul}, we looked at \( 3,034 \) ohnolog categories that have only been duplicated once (ohnolog gene pairs). Of these \( 3,034 \) ohnologs \cite{Howe2013-ul}, \( 2,107 \) were not expressed in our dataset. However the remaining 927 pairs can be divided into the following three major groups: (1) expression of ohnologs is mutually exclusive in individual cells (n = 177) (Figures \ref{fig:ohnologs}A and \ref{fig:ohnologs}B. In this group, the expression of any one ohnolog appeared to be an independent event with an equal probability of happening. This suggests selective activation or silencing of these ohnologs in individual cells; (2) only one of the ohnologs is expressed in all cells (n = 430), Figures \ref{fig:ohnologs}A and \ref{fig:ohnologs}B, and (3) both ohnologs are equally expressed in all cells (n = 218), (Figures \ref{fig:ohnologs}A and \ref{fig:ohnologs}B. No patterns of ohnolog use over pseudotime were observed.

\section{Discussion}

Here we show the power of single cell transcriptome analysis to decipher the kinetics of hematopoietic lineage development. We ordered cd41 cells by their progression through differentiation based on gene expression profiles. Our analysis illustrates the continual nature of this process, where cells progressively transit through five transcriptional states that result in the generation of mature thrombocytes.

Interestingly, myeloid lineage-affiliated genes were largely absent across all cells, suggesting direct HSC to thrombocyte-erythroid progenitor transition. The model of hematopoiesis generated recently, using single cells from over ten hematopoietic populations in mouse, implies that the megakaryocyte-erythroid lineage is closely linked to long-term repopulating HSCs and separates early from the lympho-myeloid lineage \cite{Guo2013-si}. The identification of platelet-primed stem cells within vWf-expressing long-term HSCs further confirmed that commitment to the megakaryocyte lineage starts in the most primitive stem cell compartment \cite{Sanjuan-Pla2013-qs}. Although in our dataset vWf was not expressed in any of the identified cell populations, the low expression of some of the thrombocyte lineage-affiliated genes in cluster 1a suggests that using our sorting strategy we are possibly capturing thrombocyte-primed stem cells. Therefore, HSCs in cluster 1a may represent a biased subpopulation within the wider pool of hematopoietic stem/progenitor cells present in the zebrafish kidney. Nevertheless, the gradual transition of cells during thrombocyte lineage development that we see in our dataset (e.g., gradual changes in the total number of genes as well as the total mRNA content) suggest that we do capture a continuous spectra of cells and that the common myeloid stage is not an obligatory step during thrombopoiesis.

We also show that although each of the identified transcriptional states was characterized by substantial heterogeneity in the expression of the key lineage regulators, the underlying transcriptional program was highly coordinated. It included the simultaneous increase in the expression of genes important
for thrombocyte function and suppression of genes relevant in cell proliferation and ribosomal biogenesis. Interestingly, although the maturation of thrombocytes was completed in the kidney, they maintained a transcriptionally active state in circulation. We did not, however, detect any qualitative or quantitative difference in the gene expression between circulating and kidney-based EGFP\textsuperscript{high} thrombocytes. Surprisingly, unlike mammalian platelets, which have abundant expression of vWf, thrombocytes in zebrafish do not express vWf. Instead, our analysis suggests that other cells within the kidney marrow, such as endothelial cells (fli1:GFP positive cells), express vWf in zebrafish. Finally, we assessed use of duplicated genes during thrombopoiesis in zebrafish and identified patterns of their expression that would not be possible using a bulk transcriptomics approach.

We used single-cell RNA-seq of zebrafish kidney cells to resolve the cellular hierarchy of lineage development in the myeloid branch of hematopoiesis and propose a refined model of developmental progression of hematopoietic cells.

Our study addresses some of the basic questions of regulation of differentiation, both at the molecular and cellular levels. In this study, we focused on zebrafish thrombocyte development; however, a similar approach could be used in other systems and cell types.

\section{Experimental procedures and methods}

Before describing the computational methods related to this study, we describe the experimental procedures.

\section{Marker Gene Discovery}

To discover marker genes for the clusters of cells, we trained a random forest model for each cluster versus the rest of the cells. We used the Gini feature importance scores for each gene to order genes by how well they can distinguish a cluster from the rest of the cells. We used the ExtraTreesClassifier  \cite{Geurts2006-qg} implementation in the Python machine learning package scikit-learn (Pedregosa et al., 2011), with the parameter n\_estimators = 100,000.
              
\section{Classification of Ohnolog Gene Pairs}

We obtained the list of duplicated genes arising from the teleost-specific genome duplication event from Howe et al., 2013  \cite{Howe2013-ul}. We filtered the list to only retain pairs of genes whose IDs were present in version 77 of Ensembl. For these genes, we binarized the expression to ‘‘expressed’’ or ‘‘not ex- pressed’’ in each cell based on whether the TPM was greater than 1. Using these binary values, for each Ohnolog pair we counted cells expressing either member of the pair, both members of the pair, or none of the members in the pair. Ohnolog pairs in which none of the members were expressed in more than 300 cells and were annotated as ‘‘Not expressed.’’ We defined a value ‘‘both\_min\_diff’’ as the difference between the smallest number of cells ex- pressing only one of the members in a pair, and the number of cells expressing both members of the pair. Ohnolog pairs with a ‘‘both\_min\_diff’’-value larger than 15 were annotated as ‘‘XOR Ohnologs.’’ To identify Ohnolog pairs in which only one member was used, we looked at the difference between the largest number of cells using one member compared to the largest number of cells using the other member. If this difference was larger than 60 cells, the Ohnolog pair was considered a ‘‘Single Ohnolog.’’ The remaining cells were dubbed ‘‘Mixed Ohnologs,’’ meaning cells with a mixture of both mem- bers of a pair.

\section{The \name{Pseudotime} model} \label{sec:pseudotime-model}

\subsection{Gaussian Process Regression}

The main goal of this study is to model temporal transitions.  We use the Gaussian process (GP) framework, thereby casting this problem as non-parametric regression. Let us begin by assuming that the developmental time $ t $ for each cell we observe is known. Then, the output $ y_g $ (i.e. expression of gene $ g $) is modelled as a continuous function of the input $ t $ (i.e. developmental progression)
\begin{equation}
\label{eq:gpregression}
y_g = f(t) + \varepsilon,
\end{equation}
where
$$ p(\varepsilon) = \mathcal{N}(0, \sigma^2) $$ is Gaussian distributed residual noise and $ f(t) $ denotes the unknown regression function. In this work $ y_g $ is considered to be an $ N $-dimensional vector of $ N $ cells with observed expression of the gene $ g $. We denote the expression of $ g $ in an individual cell $ n $ as $ [y_g]_n $.

A GP can be interpreted as a function-valued prior on the elements of $ f $, which is defined by a covariance function that in turn is parametrized by the input (developmental time) $ t $:
\[ \text{cov}(f(t_{n_1}), f(t_{n_2})) = k(t_n, t_{n_2}). \]
The \textit{covariance function} $ k(t_{n_1}, t_{n_2}) $ encodes prior assumptions on the smoothness and lengthscales of the function $ f(t) $. The most widely used covariance function is the Squared Exponential (SE) covariance function,
\begin{equation}
\label{eq:se-kernel}
k(t_{n_1}, t_{n_2}) = 
\sigma_{\text{SE}}^2 \exp \left( - \frac{|t_{n_1} - t_{n_2}|^2}{2l_\text{SE}^2} \right),
\end{equation}
and this is the covariance function we will generally be used in this work.  This covariance has the hyperparameters $ \mathbf{\theta} = (\sigma^2_{\text{SE}}, l^2_\text{SE}) $, which parametrize the amplitude ( $\sigma_{\text{SE}}^2 $) and the lengthscale ($ l^2_{\text{SE}} $) of functions under the prior. Throughout the remainder of the text we will omit the hyperparameters from equations for the sake of brevity. Note that there is a whole compendium of valid covariance functions, which can also be combined using sum or multiplication; see \cite{Williams2006-kb} for an overview.

We write that a function $ f $ is \textit{Gaussian Process distributed} by
\[ f(t) \sim \mathcal{GP}(0, k(t_{n_1}, t_{n_2})). \]

This prior on the function $ f $ can be linked to the finite observed data using a Gaussian likelihood:
\[ p(y_g | f) = \prod_{n=1}^N \mathcal{N} ([y_g]_n | f_{n}, \sigma^2). \]
Together with the prior on the corresponding (finite) elements of $ f $,
\[ p(f) = \mathcal{N}(f | 0, \bm{K}_t), \]
this results in the marginal likelihood
\[ p(y_g | t) = \mathcal{N}(y_g | 0, \bm{K}_t + \sigma^2 \cdot I). \]
Here $ \bm{K}_t $ is an $ N \times N $ matrix of pairwise evaluations of the covariance functions at the observed times $ t $. I.e.
\begin{equation}
\label{eq:covmatrix}
[\bm{K}_t]_{n, m} = k(t_n, t_m).
\end{equation}

By considering the joint distribution of the observed data $ y_g $ and an unseen function value $ f(t_\star) $, it is possible to derive the predictive distribution for $ f(t_\star) $:
\[
p(f(t_\star) | t, y_g, t_\star) = \mathcal{N}\left( \overline{f(t_\star)}, k(t_\star) \right),
\]
where
\begin{align*}
\overline{f(t_\star)} & = k(t_\star, t) [\bm{K}_t + \sigma^2 \cdot I]^{-1} y_g, \text{ and} \\
k(t_\star) & = k(t_\star, t_\star) - k(t_\star, t)[\bm{K}_t + \sigma^2 \cdot I]^{-1} k(t, t_\star).
\end{align*}
For a full review on Gaussian Processes, see \cite{Williams2006-kb}.

So far, we have only described Gaussian Process Regression for expression $ y_g $ of a single gene $ g $. If we consider a collection of $ G $ genes $ \{1, \ldots, G\} $, their expression can be modelled together by
\begin{equation}
\label{eq:multigp}
(y_{1}, \ldots, y_{G}) = (f^{1}(t), \ldots, f^{G}(t)) + (\varepsilon, \ldots, \varepsilon).
\end{equation}
We use $ Y $ to compactly denote the $ N \times G $ expression matrix of cells $ \times $ genes, where
\[ Y_{n, g} = [y_{g}]_n. \]
The assumption that all genes are governed by similar functional relationships with $ t $ means we place the same GP prior (with shared covariance function):
\begin{equation}
\label{eq:multigp_posterior}
p(Y | t) = \prod_{g = 1}^G p(y_{g} | t) = \prod_{g = 1}^G \mathcal{N}(y_{g} | 0, \bm{K}_{t} + \sigma^2 \cdot I).
\end{equation}
In the next section we will see the usefulness of considering multiple genes at once.

\subsection{Pseudotime inference by Bayesian GPLVM}

The Gaussian Process regression framework described above assumes we know the time $ t $ of each cell. While many scRNAseq experiments record data over some time course, these snapshots sample cells from a population where responses are asynchronous. Each cell has reached a certain stage in the differentiation process under investigation, which we do not observe directly. The progress in to this process is referred to as \textit{pseudotime}. We can however infer this from the data. In the Gaussian Process Latent Variable Model (GPLVM) \cite{Lawrence2006-it}, we use the multiple output case of Gaussian Process regression (equation \ref{eq:multigp}), but consider the values of $ t $ to be parameters which we wish to infer.

The joint probability of the GPLVM is
\[ p(Y, t) = p(Y | t) \cdot p(t), \]
where $ p(Y | t) $ is defined in equation \ref{eq:multigp_posterior}, and the prior $ p(t) $ is such that for cell $ n $,
\[ p(t_n) = \mathcal{N}(0, 1). \]

The objective of \textit{Bayesian} GPLVM \cite{Titsias2010-hq}, is to find the posterior probability distribution $ p(t | Y) \propto p(Y | t) \cdot p(t) $. This is intrugh, due to the $ t $ values appearing non-linearly in the matrix inverse $ [\bm{K}_t + \sigma^2 \cdot I]^{-1} $. In \cite{Titsias2010-hq}, a lower bound to the marginal likelihood is calculated by estimating the  posterior $ p(t | Y) $ by a variational distribution $ q(t). $ The distribution
\[ q(t) = \prod_{n=1}^N \mathcal{N} (t_n | \mu_n, S_n ) \]
is described in that paper, and Bayesian training of the model to maximise this lower bound. This is the method we use.

\subsection{Downstream pseudotime analysis}

After a \name{pseudotime} has been found, data can be processed as you would analyze time-courses.

\subsection{Identifying dynamic genes}

We treated the pseudotime progression order of the cells as a time series, and for each gene trained two Gaussian processes (GPs): one with a radial basis function (RBF) kernel (which can model change over time) and one with a con- stant kernel (which assumes that the expression of a gene does not change over time). After optimizing parameters for both models, we filtered the genes by the ratios of the likelihoods of the models. If the RBF kernel GP has a higher likelihood than the constant kernel GP, we can conclude that the gene in question has expression that is dynamic in time.

\subsection{Clustering genes in to expression trends}

Once we had identified genes that were pseudotime-dependent, we applied the mixtures of hierarchical Gaussian processes model to identify groupings of genes with similar pseudotime expression patterns as described in \cite{Hensman2015-op}. In this setting, we assume the data is generated by
\[
    y_g = f_{c, g} ( g_c(t) ) + \varepsilon,
\]
where all \(  f_{c, g} \) and \( g_c \) are Gassuain process distributed:
\begin{align*}
    g_c(t) & \sim \mathcal{GP}(0, k_g(t_1, t_2)), \\
    f_{c, g}(t) & \sim \mathcal{GP}(g_c(t), k_f(t_1, t_2)).
\end{align*}
 This model uses the structure of the data which can be expressed in terms of the covariance function
\[
    k((t_1, c_1), (t_2, c_2)) = k_f(t_1, t_2) + [c_1 = c_2] \cdot k_c(t_1, t_2),
\]
(using the Iverson bracket notation for indicators). The cluster assignments \( c \) are not known, and need to be learned. We use the implementation from \cite{Hensman2015-op} for this.

 All functional enrichment analysis was performed with the gProfiler \cite{Reimand2011-ht} web service with the standard gene list as background.
